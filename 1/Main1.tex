		%iffalse
\let\negmedspace\undefined
\let\negthickspace\undefined
\documentclass[journal,12pt,twocolumn]{IEEEtran}
\usepackage{cite}
\usepackage{amsmath,amssymb,amsfonts,amsthm}
\usepackage{algorithmic}
\usepackage{graphicx}
\usepackage{textcomp}
\usepackage{xcolor}
\usepackage{txfonts}
\usepackage{listings}
\usepackage{enumitem}
\usepackage{mathtools}
\usepackage{gensymb}
\usepackage{comment}
\usepackage[breaklinks=true]{hyperref}
\usepackage{tkz-euclide} 
\usepackage{listings}
\usepackage{gvv}                                        
%\def\inputGnumericTable{}                                 
\usepackage[latin1]{inputenc}                                
\usepackage{color}                                            
\usepackage{array}                                            
\usepackage{longtable}                                       
\usepackage{calc}                                             
\usepackage{multirow}                                         
\usepackage{hhline}                                           
\usepackage{ifthen}                                           
\usepackage{lscape}
\usepackage{tabularx}
\usepackage{array}
\usepackage{float}
\usepackage{multicol}
\newcounter {sectioicolsn}

\newtheorem{theorem}{Theorem}[section]
\newtheorem{problem}{Problem}
\newtheorem{proposition}{Proposition}[section]
\newtheorem{lemma}{Lemma}[section]
\newtheorem{corollary}[theorem]{Corollary}
\newtheorem{example}{Example}[section]
\newtheorem{definition}[problem]{Definition}
\newcommand{\BEQA}{\begin{eqnarray}}
\newcommand{\EEQA}{\end{eqnarray}}
\newcommand{\define}{\stackrel{\triangle}{=}}
\theoremstyle{remark}
\newtheorem{rem}{Remark}
% Marks the beginning of the document
\begin{document}
\bibliographystyle{IEEEtran}
\vspace{3cm}
\title{Chapter-11 Section-A }
\author{ee24btech11050 - Rahul Pothuri}
\maketitle
\newpage
\bigskip
\renewcommand{\thefigure}{\theenumi}
\renewcommand{\thetable}{\theenumi}
\begin{enumerate}[start=26]
\item %26
Let $[x]$ be the greatest integer less than or equal to $x$. Then, at which of the following points the function
$f\brak{x} = x \cos\brak{\pi\brak{x + [x]}}$ is discontinuous? 
 \hfill  (JEE Adv.2017) \\
\begin{enumerate} 
\begin{multicols}{2}
\item $x=-1$
\item $x=0$
\end{multicols}
\begin{multicols}{2}
\item $x=1$
\item $x=2$
\end{multicols}
\end{enumerate}
\item %27
 Let $f\brak{x}=\frac{1-x \brak{1+\abs{1-x}}}{\abs{1-x}}\cos\left(\frac{1}{\abs{1-x}}\right)$ for $x\neq1$ .Then  
\hfill (JEE Adv.2017) \\ 
\begin{enumerate}
\item $\lim_{x\to1^-} f\brak{x}=0$ 
\item $\lim_{x\to1^-}$does not exist
\item $\lim_{x\to1^+} f\brak{x}=0$ 
\item $\lim_{x\to1^+}$does not exist 
\end{enumerate}
\item %28
Let $f:\mathbb{R} \to \mathbb{R}$ $g:\mathbb{R} \to \mathbb{R}$ be two non-differentiable functions. If $f'\brak{x}=(e^{\brak{f\brak{x}-g\brak{x}}})g'\brak{x}$ for all $x\in \mathbb{R}$ ,and $f\brak{1}=g\brak{2}=1$, then which of the following statement \brak{s} is \brak{are} TRUE? 
\hfill  (JEE Adv.2018) \\ 
\begin{enumerate}
\begin{multicols}{2}
\item $f\brak{2}<1-\log_e 2$ 
\item $f\brak{2}>1-\log_e 2$ 
\end{multicols}
\begin{multicols}{2}
\item $g\brak{2}>1-\log_e 2$ 
\item $g\brak{2}<1-\log_e 2$ 
\end{multicols}
\end{enumerate}
\item %29
Let$f:\mathbb{R} \to \mathbb{R}$ given by \\
$f\brak{x} = 
\begin{cases} 
x^5+5x^4+10x^3+10x^2+3x+1, & \text{if } x < 0; \\
x^2-x+1, & \text{if } 0 \leq x <1; \\
\frac{2}{3}x^3-4x^2=7x-\frac{8}{3} & \text{if } 1 \leq x <3; \\
\brak{x-2}\log_e\brak{x-2}-x=\frac{10}{3},& \text{if } x \geq 3
\end{cases}$ \\
Then which of the following option is/are correct?
\hfill  (JEE Adv.2019) \\
\begin{enumerate}
\item $f'$ has a local maximum at at x=1 
\item $f$ is increasing on $\brak{-\infty , 0}$
\item $f'$ is NOT differentiable at $x=1$
\item $f$is onto 
\end{enumerate} 
 \item %30
 Let $f:\mathbb{R} \to \mathbb{R}$ be a function. We say that $f$ has \\
\textbf{PROPERTY 1} if $\lim_{h\to0}\frac{f\brak{h}-f\brak{0}}{\sqrt{\abs{h}}}$ exists and is finite, and \\
\textbf{PROPERTY 2} if $\lim_{h\to0}\frac{f\brak{h}-\brak{0}}{h^2}$ exists and is finite \\
Then which of the following options is/are correct? 
\hfill (JEE Adv.2019) \\
\begin{enumerate}
\item $f\brak{x}=x^{\frac{2}{3}}$ has \textbf{PROPERTY 1} 
\item $f\brak{x}=\sin{x}$ has \textbf{PROPERTY 2} 
\item $\brak{x}=\abs{x}$ has \textbf{PROPERTY 1} 
\item $\brak{x}=x\abs{x}$ has \textbf{PROPERTY 2} 
\end{enumerate}
\end{enumerate}
\end{document}
