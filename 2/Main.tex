%iffalse
\let\negmedspace\undefined
\let\negthickspace\undefined
\documentclass[journal,12pt,twocolumn]{IEEEtran}
\usepackage{cite}
\usepackage{amsmath,amssymb,amsfonts,amsthm}
\usepackage{algorithmic}
\usepackage{graphicx}
\usepackage{textcomp}
\usepackage{xcolor}
\usepackage{txfonts}
\usepackage{listings}
\usepackage{enumitem}
\usepackage{mathtools}
\usepackage{gensymb}
\usepackage{comment}
\usepackage[breaklinks=true]{hyperref}
\usepackage{tkz-euclide} 
\usepackage{listings}
\usepackage{gvv}                                        
%\def\inputGnumericTable{}                                 
\usepackage[latin1]{inputenc}                                
\usepackage{color}                                            
\usepackage{array}                                            
\usepackage{longtable}                                       
\usepackage{calc}                                             
\usepackage{multirow}                                         
\usepackage{hhline}                                           
\usepackage{ifthen}                                           
\usepackage{lscape}
\usepackage{tabularx}
\usepackage{array}
\usepackage{float}
\usepackage{multicol}
\newcounter {sectioicolsn}
\newtheorem{theorem}{Theorem}[section]
\newtheorem{problem}{Problem}
\newtheorem{proposition}{Proposition}[section]
\newtheorem{lemma}{Lemma}[section]
\newtheorem{corollary}[theorem]{Corollary}
\newtheorem{example}{Example}[section]
\newtheorem{definition}[problem]{Definition}
\newcommand{\BEQA}{\begin{eqnarray}}
\newcommand{\EEQA}{\end{eqnarray}}
\newcommand{\define}{\stackrel{\triangle}{=}}
\theoremstyle{remark}
\newtheorem{rem}{Remark}
% Marks the beginning of the document
\begin{document}
\bibliographystyle{IEEEtran}
\vspace{3cm}
\title{Chapter-11 Section-A}
\author{ee24btech11050 - Rahul Pothuri}
\maketitle
\newpage
\bigskip
\renewcommand{\thefigure}{\theenumi}
\renewcommand{\thetable}{\theenumi}
\begin{enumerate}[start=1]
\item 
Evaluate $\lim_{x \to a}\frac{\sqrt{a+2x}-\sqrt{3x}}{\sqrt{3a+x}-2\sqrt{x}} , \brak{a\neq 0}$
\hfill(1978)\\
\item
$f\brak{x}$ is the integral of $\frac{2\sin{x}\sin{2x}}{x^3}$, $x\neq0$,find $\lim_{x\to0}f'\brak{x}$ 
  \hfill(1979) \\
\item
Evaluate: \\$\lim_{h\to0}\frac{\brak{a+h}^2\sin{\brak{a+h}}-a^2\sin{a}}{h}$ 
\hfill(1980)\\
\item
Let $f\brak{x+y}=f\brak{x}+f\brak{y}$ for all $x$ and $y$.\\If the function $f\brak{x}$is continuous at $x=0$,then show that $f\brak{x}$ is continuous at all $x$. 
  \hfill(1981 - 2Marks) \\
\item
Use the formula $\lim_{x\to0}\frac{a^x-1}{x}=lna$ to find $\lim_{x\to0}\frac{2^x-1}{(1+x)^{\frac{1}{2}}-1}$ \hfill (1982 - 2 Marks) \\
\item
Let 
$
f\brak{x} = 
\begin{cases} 
$1+x$  \quad 0\leq x\leq 2 \\
$3-x$  \quad 2\leq x \leq 3
\end{cases}
$ \\
Determine the form of $g(\brak{x}=f\brak{x}$ and hence find the points of discontinuity of $g$,\\if any 
\hfill(1983 - 2 Marks) \\  
\item
Let $
f\brak{x} = 
\begin{cases} 
\frac{x^2}{2} \quad \quad \quad \quad  \quad,0\leq x\leq 1 \\
2x^2-3x+\frac{3}{2}  \quad,1\leq x \leq 2
\end{cases}
$ \\
Discuss the continuity of $f$,$f'$ and $f''$ on\\ \sbrak{0,2}.
 \hfill(1983 - 2 Marks) \\
\item
Let $f\brak{x}=x^3-x^2+x+1$ and \\ $g\brak{x}=$ 
       $ \begin{cases}
        max{f(t);0\leq t\leq1} \\   
        3-x  \quad 0\leq t\leq2 \\
        \end{cases}$
Discuss the continuity and differentiability of the function $g\brak{x}$ in the interval\\ \brak{0,2} 
 \hfill(1985 - 5 Marks) \\
\item
Let $f\brak{x}$ be defined in the interval \sbrak{-2,2}such that \\
$
$f\brak{x}$ = 
\begin{cases} 
$-1$ ,-2\leq x\leq 0 \\
$x-1$ ,0<x \leq 2
\end{cases}
$ \\
and $g\brak{x}=f\brak{\abs{x}}+\abs{f\brak{x}}$ 
Test the differentiability of $g\brak{x}$ in $\brak{-2,2}$.
 \hfill{(1986 - 5 Marks)} \\
\item
Let $f\brak{x}$ be a continuous and $g\brak{x}$ be a discontinuous 
function.Prove that $f\brak{x}+g\brak{x}$ is a discontinuous function.
   \hfill{(1987 -2Marks)}
\end{enumerate}
\end{document}
